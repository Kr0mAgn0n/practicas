\chapter{Introducción}

Esta monografía tiene como objetivo aplicar los resultados básicos de la Teoría Ergódica a un caso en particular como la rotación irracional de círculo.

El principal resultado que se quiere exponer es que en la rotación irracional del círculo, la medida de Lebesgue es la única medida ergódica e invariante con respecto a la función que describe la dinámica de la rotación irracional de círculo.

En el primer capítulo se exponen los resultados de Teoría de la Medida que serán usados en la demostraciones.

En el segundo capítulo se exponen los resultados de la Teoría Ergódica siendo el Teorema Ergódico de Birkhoff el resultado principal. Veremos posteriormente que el teorema antes mencionado será usado para demostrar de manera directa que la órbita de la dinámica es densa en $S^1$".

En el tercer capítulo se expondrá el resultado principal de la monografía con algunos resultados necesarios para desarrollarlo.

Por último está el Anexo que fue hecho para justificar un paso en una demostración del tercer capítulo cuya demostración no es trivial.

Para terminar con la introducción hay que mencionar que este trabajo está basado en el curso del International Centre for Theoretical Physics (ICTP) de Teoría Ergódica. Los videos se pueden consultar en \cite{youtube}.

También se usaron varias referencias en internet que están citadas al final de la monografía.

En el caso de texto, se uso principalmente \cite{bartle} para los conceptos de Teoría de la Medida.