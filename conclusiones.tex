\chapter*{Conclusiones}
\addcontentsline{toc}{chapter}{Conclusiones}

Los resultados del último capítulo dicen que la única medida ergódica e invariante con respecto a la función que describe el sistema dinámico de la rotación irracional es la medida de Lebesgue.

Este resultado nos deja con una única medida de probabilidad con la cuál analizar las propiedades ergódicas de la rotación irracional y además nos lleva a afirmar que existe solo una medida de probabilidad invariante ya que esta única medida de probabilidad ergódica es un punto extremo del conjunto de medidas de probabilidad invariantes que es convexo.

Por otro lado, con el teorema \ref{dist_uniforme} se observa que la órbita está uniformente distribuída.

Esto hace que la rotación irracional sea un ejemplo destacado para la aplicación directa del Teorema Ergódico de Birkhoff que es el resultado más importante del Capítulo 3 de esta monografía.