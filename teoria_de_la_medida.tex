\section{Teoría de la Medida}

\begin{teorema}
	Sea $M$ un espacio métrico compacto. $\mathcal{C}(M,\mathbb{C})$ es denso en $L^2$.
\end{teorema}

\cite{upennpdfL2}

\begin{definicion}
	Una familia de funciones $\{f_n\}$ \textit{genera} $L^2(S^1,\mathbb{C})$ si el conjunto de sus combinaciones lineales es denso en $L^2(S^1,\mathbb{C})$.
\end{definicion}

\begin{definicion}
	Una \textit{base ortonormal} de $L^2(S^1,\mathbb{C})$ es una familia ortonormal que genera todo el espacio.
\end{definicion}

\begin{lema}
	La familia $\{e_n\}_{n \geq 1}$ es una base ortonormal de $L^2(S^1,\mathbb{C})$ si y solo si la única función $f$ que es ortogonal a cada miembro de la familia es $f \equiv 0$.
\end{lema}

\begin{teorema}
	El subespacio generado por $\{e^{2\pi inx}\}_{n \geq 1}$ es un conjunto denso en $L^2([0,1),\mathbb{C})$.
\end{teorema}

\begin{proof}
	Vemos que
	
	\begin{align}
		\langle e^{2\pi inx},e^{2\pi imx} \rangle &= \int_{0}^{1} e^{2\pi inx} e^{-2\pi imx} dx\\
		&= \int_{0}^{1} e^{2\pi i(n-m)x} dx
	\end{align}
	
	De allí tenemos que
	
	\begin{equation}
		\langle e^{2\pi inx},e^{2\pi imx} \rangle = \begin{cases}
			1, \quad m=n\\			
			0,\quad m \neq n
		\end{cases}
	\end{equation}
	
	Para demostrar que el subespacio generado por $\{e^{2\pi inx}\}_{n \geq 1}$ es denso hay que demostrar que
	
	\begin{equation}
		f(x) = \sum_{k=0}^{\infty} \langle f,e^{2\pi ikx} \rangle e^{2\pi ikx}, \quad \forall x \in [0,1)
	\end{equation}
	
	para cualquier $f \in L^2([0,1),\mathbb{C})$. Denotemos a $e^{2\pi ikx}$ como $e_k$.
	
	\begin{equation}
		\left\| f - \sum_{k=0}^{n} \langle f,e_k \rangle e_k \right\|^2 = \left\langle f - \sum_{k=0}^{n} \langle f,e_k \rangle e_k, f - \sum_{k=0}^{n} \langle f,e_k \rangle e_k \right\rangle
	\end{equation}
	
	Resolviendo la ecuación con las propiedades del producto interno tenemos
	
	\begin{equation}
		\left\| f - \sum_{k=0}^{n} \langle f,e_k \rangle e_k \right\|^2 = \|f\|^2 - \sum_{k=0}^{n} |\langle f,e_k \rangle|^2
	\end{equation}
	
	Como esta igualdad es no negativa, vemos que
	
	\begin{equation}
		\|f\|^2 \geq \sum_{k=0}^{n} |\langle f,e_k \rangle|^2
	\end{equation}
	
	Al final tenemos que la sucesión $\sum_{k=0}^{n} |\langle f,e_k \rangle|^2$ es monótona creciente y acotada por lo tanto es convergente y converge a su supremo. 
\end{proof}