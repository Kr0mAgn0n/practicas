\section{Teoría de la Medida}

\begin{definicion}
	Sea $\mu$ una medida en $(X,\Sigma)$.Se dice que la medida $\nu$ es \textit{absolutamente continua} con respecto a $\mu$ si
	
	\begin{equation}
		\forall A \in \Sigma, \mu A = 0 \Rightarrow \nu A = 0
	\end{equation}
	
	La notación $\nu \ll \mu$ indica que $\nu$ es absolutamente continua con respecto a $\mu$.
\end{definicion}

\begin{teorema}[Radon-Nikodyn]\label{radon-nikodyn}
	Sea $(X,\Sigma,\mu)$ un espacio de probabilidad. Sea $\nu$ una medida definida en $\Sigma$ con $\nu \ll \mu$. Entonces existe una función no negativa medible $f$ tal que
	
	\begin{equation}
		\nu A = \int_A f d\mu, \forall A \in \Sigma
	\end{equation}
	
	Además $f$ es única en el sentido de que si $g$ es una función que cumple la misma propiedad, entonces $f = g$ $\mu$-c.t.p.
\end{teorema}

\begin{definicion}
	Sea $\Sigma' \subset \Sigma$ una sub-$\sigma$-álgebra, $f \in L^1(X,\Sigma,\mu)$ con $f \geq 0$ en $\mu$-c.t.p y $\nu$ una medida definida en $\Sigma'$ con
	
	\begin{equation}
		\nu A = \int_A f d\mu, \forall A \in \Sigma'
	\end{equation}
	
	Como $\nu \ll \mu|_{\Sigma'}$, por el teorema \ref*{radon-nikodyn}, se tiene que existe una función $\Sigma'$-medible no negativa en $\mu$-c.t.p. que se denotará $E(f|\Sigma')$ tal que
	
	\begin{equation}
		\nu A = \int_A E(f|\Sigma') d\mu, \forall A \in \Sigma'
	\end{equation}
	
	A la función $E(f|\Sigma')$ se le llama \textit{esperanza condicionada de $f$ con respecto a $\Sigma'$}. 
\end{definicion}

Hay que observar que $E(f|\Sigma')$ está definida para $f \geq 0$. Este concepto se puede extender para funciones $f \in L^1(X,\Sigma,\mu)$ en general. Se define $E(f|\Sigma')$ para $f \in L^1(X,\Sigma,\mu)$ de la siguiente forma

\begin{equation}
	E(f|\Sigma') = E(f_+|\Sigma')- E(f_-,\Sigma')
\end{equation}

\begin{definicion}\label{trivial_sigma-algebra}
	Sea $(X,\Sigma,\mu)$ un espacio de probabilidad. Se define $N \subset \Sigma$ como
	
	\begin{equation}
	N = \{ A \in \Sigma: \mu A = 0 \vee \mu A = 1 \}
	\end{equation}
\end{definicion}

Se puede demostrar que $N$ es un $\sigma$-álgebra. $\emptyset \in N$ porque $\mu \emptyset = 0$. Sea $A \in N$ entonces $\mu A = 0 \vee \mu A = 1$. Como $\mu(X \setminus A) = 1 - \mu A$ entonces se tiene que $\mu(X \setminus A) = 0 \vee \mu(X \setminus A) = 1$. Con esto se tiene que $X \setminus A \in N$. Por último, sea $\{A_n\}_{n \geq 1} \subset N$.  Si suponemos que $0 < \mu(\cup_{n \geq 1} A_n) \leq \sum_{n \geq 1} \mu A_n$ entonces $\forall n \geq 1, 0 < \mu A_n$. Con esto se tiene que $\forall n \geq 1, \mu A_n = 1$. Luego se tiene $\mu(X \setminus \cup_{n \geq 1} A_n) \leq \mu(X \setminus \cap_{n \geq 1} A_n) = \mu(\cup_{n \geq 1} (X \setminus A_n)) \leq \sum_{n \geq 1} \mu(X \setminus A_n) = 0$. Por lo tanto, si suponemos que $0 < \mu(\cup_{n \geq 1} A_n)$  se llega a que $\mu(X \setminus \cup_{n \geq 1} A_n) = 0$ y con esto a $\mu(\cup_{n \geq 1} A_n) = 1$. Por lo tanto $N$ es un $\sigma$-álgebra.

\begin{propiedad}\label{esperanza_en_sigma-algebra_trivial}
	Sea $(X,\Sigma,\mu)$ un espacio de probabilidad, $f \in L^1(X,\Sigma,\mu)$ y $N'$ un sub-$\sigma$-álgebra de $N$. Entonces
	
	\begin{equation}
	\int_A E(f|N') d\mu = E(f|N'), \forall A \in N
	\end{equation}
\end{propiedad}

\begin{proof}
	Supongamos que $\int_A E(f|N') d\mu > E(f|N'), \forall A \in N'$. Luego $\int_A E(f|N') d\mu - E(f|N') > 0$. Integrando tenemos
	
	\begin{equation}
	\int_A \left( \int_A E(f|N')d\mu - E(f|N') \right) d\mu > 0
	\end{equation}
	
	Como $A \in N'$ entonces $\mu A = 0 \vee \mu A = 1$. Pero si $\mu A = 0$ entonces $0>0$. Contradicción. Por otro lado, si $\mu A = 1$ también se obtendría $0>0$. Se llegaría a la misma contradicción si asumiéramos al inicio que $\int_A E(f|N') d\mu < E(f|N'), \forall A \in N'$. Por lo tanto, $\int_A E(f|N') d\mu = E(f|N')$.
\end{proof}