\chapter{Conceptos preliminares de Teoría de la Medida}

En este capítulo se expondrán los conceptos necesarios de Teoría de la Medida que van a ser usados en los siguientes capítulos. Los conceptos básicos serán comentados y los importantes serán resaltados. la referencia usada en esta sección es \cite{bartle}.

\begin{teorema}[Teorema de la Convergencia Monótona]\label{convergencia_monotona}
	Sean $(X,\Sigma,\mu)$ un espacio medible, los $f_n: X \rightarrow \mathbb{R}$ y $f: X \rightarrow \mathbb{R}$. Si $(f_n)$ es una sucesión creciente de funciones medibles no negativas que converge a $f$ entonces
	\begin{equation}
		\int f d\mu = \lim \int f_n d\mu
	\end{equation}
\end{teorema}

\begin{proof}
	Es claro que $f$ es una función medible. Puesto que $f_n \leq f_{n+1} \leq f$ entonces
	\begin{equation}
		\int f_n d\mu \leq \int f_{n+1} d\mu \leq \int f d\mu
	\end{equation}
	
	para todo $n$. Entonces
	\begin{equation}
		\lim \int f_n d\mu \leq \int f d\mu
	\end{equation}
	
	Por otro lado, sea $0 < \alpha < 1$ cualquiera y $\varphi: X \rightarrow \mathbb{R}$ una función simple cualquiera tal que $0 \leq \varphi \leq f$. Se define
	\begin{equation}
		A_n = \{ x \in X: f_n(x) \geq \alpha \varphi(x) \}
	\end{equation}
	
	Se puede observar que $A_n \subset A_{n+1}$ y que $A_n \rightarrow X$ cuando $n \rightarrow \infty$. Luego
	\begin{equation}
		\int_{A_n} \alpha \varphi d\mu \leq \int_{A_n} f_n d\mu \leq \int f_n d\mu
	\end{equation}
	
	Como $A_n \rightarrow X$ entonces se puede afirmar que
	\begin{equation}
		\int \varphi d\mu = \lim \int_{A_n} \varphi d\mu
	\end{equation}
	
	Luego en el penúltimo resultado procedemos a tomar limite
	\begin{equation}
		\int \alpha \varphi d\mu \leq \lim \int f_n d\mu
	\end{equation}
	
	Definamos $B=\{\varphi \text{ simple}: \varphi \leq f \}$ y $C = \{ \alpha \varphi: \varphi \in B \}$ con $\alpha \in (0,1)$ fijo arbitrario. Se observa que $B=C$. Luego $\alpha \varphi$ representa todas las funciones simples no mayores que $f$. Entonces, tomando supremo
	\begin{equation}
		\int f d\mu \leq \lim \int f_n d\mu
	\end{equation}	
	
	Por lo tanto, $\int f d\mu = \lim \int f_n d\mu$.
\end{proof}

\begin{teorema}[Lema de Fatou]\label{fatou}
	Sea $(f_n)$ una sucesión de funciones medibles no negativas. Entonces
	\begin{equation}
		\int (\liminf f_n) d\mu \leq \liminf \int f_n d\mu
	\end{equation}
\end{teorema}

\begin{proof}
	Sea $g_m = \inf \{ f_m,f_{m+1},\ldots \}$. Se observa que $g_m \leq f_n$ siempre que $m \leq n$. Entonces
	\begin{equation}
		\int g_m d\mu \leq \int f_n d\mu, \qquad m \leq n
	\end{equation}
	
	Luego se tiene
	\begin{equation}
		\int g_m d\mu \leq \liminf 	\int f_n d\mu
	\end{equation}
	
	Como la sucesión $(g_m)$ es creciente y converge a $\liminf f_n$, entonces, por Teorema de Convergencia Monótona \ref{convergencia_monotona}
	\begin{align}
		\int (\liminf f_n) d\mu &= \int g_m d\mu\\
		&\leq \liminf \int f_n d\mu
	\end{align}
\end{proof}

\begin{teorema}[Teorema de Convergencia Dominada de Lebesgue]\label{convergencia_dominada}
	Sea $(X,\Sigma,\mu)$ un espacio de medible y $f_n,g,f: X \rightarrow \mathbb{R}$. Sea $(f_n)$ una sucesión de funciones integrables que convergente puntualmente en casi todo punto  a una función medible $g$. Si existe una función integrable $g$ tal que $|f_n| \leq g$ para todo $n$, entonces $f$ es integrable y
	\begin{equation}
		\int f d\mu = \lim_{n \rightarrow \infty} \int f_n d\mu
	\end{equation}
\end{teorema}

\begin{proof}
	Como $|f_n| \leq g$ entonces $g + f_n \geq 0$. Aplicando el lema de Fatou \ref{fatou} tenemos
	\begin{align}
		\int g d\mu + \int f d\mu &= \int (g+f) d\mu \leq \liminf \int (g+f_n) d\mu\\
		&= \liminf \left( \int g d\mu + \int f_n d\mu \right)\\
		&= \int g d\mu + \liminf \int f_n d\mu
	\end{align}
	
	Luego se tiene que
	\begin{equation}
		\int f d\mu \leq \liminf \int f_n d\mu
	\end{equation}
	
	Por otro lado, también se cumple que $g - f_n \geq 0$. Aplicando de nuevo el lema de Fatou se tiene
	\begin{align}
		\int g d\mu - \int f d\mu &= \int (g-f) d\mu \leq \liminf \int (g - f_n) d\mu\\
		&= \int g d\mu - \limsup \int f_n d\mu
	\end{align}
	
	De la ecuación anterior se concluye que
	\begin{equation}
		\limsup \int f_n d\mu \leq \int f d\mu
	\end{equation}
	
	Si juntamos este resultado con el resultado anterior se observa que
	\begin{equation}
		\limsup \int f_n d\mu \leq \int f d\mu \leq \liminf \int f_n d\mu
	\end{equation}
	
	Por lo tanto, el límite de $\int f_n d\mu$ existe y
	\begin{equation}
		\int f d\mu = \lim \int f_n d\mu
	\end{equation}
\end{proof}

\begin{definicion}
	Sea $\mu$ una medida en $(X,\Sigma)$. Se dice que la medida $\nu$ es \textit{absolutamente continua} con respecto a $\mu$ si	
	\begin{equation}
		\forall A \in \Sigma, \mu A = 0 \Rightarrow \nu A = 0
	\end{equation}
	
	La notación $\nu \ll \mu$ indica que $\nu$ es absolutamente continua con respecto a $\mu$.
\end{definicion}

\begin{teorema}[Radon-Nikodyn]\label{radon-nikodyn}
	Sea $(X,\Sigma,\mu)$ un espacio de probabilidad. Sea $\nu$ una medida definida en $\Sigma$ con $\nu \ll \mu$. Entonces existe una función no negativa medible $f$ tal que	
	\begin{equation}
		\forall A \in \Sigma, \nu A = \int_A f d\mu
	\end{equation}
	
	Además $f$ es única en el sentido de que si $g$ es una función que cumple la misma propiedad, entonces $f = g$ $\mu$-c.t.p.
\end{teorema}

\begin{definicion}\label{esperanza_conficionada_def}
	Sea $\Sigma' \subset \Sigma$ una sub-$\sigma$-álgebra, $f \in L^1(X,\Sigma,\mu)$ con $f \geq 0$ en $\mu$-c.t.p y $\nu$ una medida definida en $\Sigma'$ con	
	\begin{equation}
		\nu A = \int_A f d\mu, \forall A \in \Sigma'
	\end{equation}
	
	Como $\nu \ll \mu|_{\Sigma'}$, por el teorema \ref*{radon-nikodyn}, se tiene que existe una única función $\Sigma'$-medible no negativa en $\mu$-c.t.p. que se denotará $E(f|\Sigma')$ tal que	
	\begin{equation}
		\nu A = \int_A E(f|\Sigma') d\mu, \forall A \in \Sigma'
	\end{equation}
	
	A la función $E(f|\Sigma')$ se le llama \textit{esperanza condicionada de $f$ con respecto a $\Sigma'$}. 
\end{definicion}

Hay que observar que $E(f|\Sigma')$ está definida para $f \geq 0$. Este concepto se puede extender para funciones $f \in L^1(X,\Sigma,\mu)$ en general. Se define $E(f|\Sigma')$ para $f \in L^1(X,\Sigma,\mu)$ de la siguiente forma
\begin{equation}
	E(f|\Sigma') = E(f_+|\Sigma')- E(f_-,\Sigma')
\end{equation}

\begin{definicion}\label{trivial_sigma-algebra}
	Sea $(X,\Sigma,\mu)$ un espacio de probabilidad. Se define $N \subset \Sigma$ como	
	\begin{equation}
	N = \{ A \in \Sigma: \mu A = 0 \vee \mu A = 1 \}
	\end{equation}
\end{definicion}

Se puede demostrar que $N$ es un $\sigma$-álgebra. $\emptyset \in N$ porque $\mu \emptyset = 0$. Sea $A \in N$ entonces $\mu A = 0 \vee \mu A = 1$. Como $\mu(X \setminus A) = 1 - \mu A$ entonces se tiene que $\mu(X \setminus A) = 0 \vee \mu(X \setminus A) = 1$. Con esto se tiene que $X \setminus A \in N$. Por último, sea $\{A_n\}_{n \geq 1} \subset N$.  Si suponemos que $0 < \mu(\cup_{n \geq 1} A_n) \leq \sum_{n \geq 1} \mu A_n$ entonces $\forall n \geq 1, 0 < \mu A_n$. Con esto se tiene que $\forall n \geq 1, \mu A_n = 1$. Luego se tiene $\mu(X \setminus \cup_{n \geq 1} A_n) \leq \mu(X \setminus \cap_{n \geq 1} A_n) = \mu(\cup_{n \geq 1} (X \setminus A_n)) \leq \sum_{n \geq 1} \mu(X \setminus A_n) = 0$. Por lo tanto, si suponemos que $0 < \mu(\cup_{n \geq 1} A_n)$  se llega a que $\mu(X \setminus \cup_{n \geq 1} A_n) = 0$ y con esto a $\mu(\cup_{n \geq 1} A_n) = 1$. Por lo tanto $N$ es un $\sigma$-álgebra.

\begin{propiedad}\label{esperanza_en_sigma-algebra_trivial}
	Sea $(X,\Sigma,\mu)$ un espacio de probabilidad, $f \in L^1(X,\Sigma,\mu)$ y $N'$ un sub-$\sigma$-álgebra de $N$. Entonces	
	\begin{equation}
	\int_A E(f|N') d\mu = E(f|N'), \forall A \in N
	\end{equation}
\end{propiedad}

\begin{proof}
	Supongamos que $\int_A E(f|N') d\mu > E(f|N'), \forall A \in N'$. Luego $\int_A E(f|N') d\mu - E(f|N') > 0$. Integrando tenemos
	
	\begin{equation}
	\int_A \left( \int_A E(f|N')d\mu - E(f|N') \right) d\mu > 0
	\end{equation}
	
	Como $A \in N'$ entonces $\mu A = 0 \vee \mu A = 1$. Pero si $\mu A = 0$ entonces $0>0$. Contradicción. Por otro lado, si $\mu A = 1$ también se obtendría $0>0$. Se llegaría a la misma contradicción si asumiéramos al inicio que $\int_A E(f|N') d\mu < E(f|N'), \forall A \in N'$. Por lo tanto, $\int_A E(f|N') d\mu = E(f|N')$.
\end{proof}