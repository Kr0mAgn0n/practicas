\chapter{Introducción a la Teoría Ergódica}

En esta sección consideraremos a $X$ como un espacio métrico compacto, a $\Sigma$ como un $\sigma$-álgebra de $X$ y $f \in \mathcal{C}(X)$.

\begin{definicion}
	Sea $(X,\Sigma,\mu)$ un espacio de probabilidad. La medida $\mu$ se dice $f$-invariante si	
	\begin{equation}
		\mu(f^{-1}A) = \mu A
	\end{equation}
	
	para cualquier $A \in \Sigma$.
\end{definicion}

\begin{definicion}
	Sea $(X,\Sigma,\mu)$ un espacio de probabilidad  con $\mu$ $f$-invariante. Se define $\Sigma_f$ como	
	\begin{equation}
		\Sigma_f = \{ A \in \Sigma: \mu(f^{-1}A) = \mu A \}
	\end{equation}
\end{definicion}

Con esta definición se puede probar $\Sigma_f$ es un $\sigma$-álgebra. Se puede observar que $\emptyset \in \Sigma_f$ por que $\mu(f^{-1} \emptyset) = \mu \emptyset = 0$. Sea $A \in \Sigma_f$. Entonces $\mu(f^{-1}(X \setminus A)) =  \mu(X \setminus f^{-1}A) = 1 - \mu(f^{-1}A) = 1 - \mu A = \mu(X \setminus A)$. Luego $X \setminus A \in \Sigma$. Por último, sea $\{A_n\}_{n \geq 1} \subset \Sigma_f$. Entonces $\mu(f^{-1}A_n \setminus A_n) = 0$. Se sigue que $\mu(f^{-1}(\cup_{n \geq 1} A_n) \setminus (\cup_{n \geq 1} A_n)) \leq \mu(\cup_{n \geq 1} (f^{-1}A_n \setminus A_n)) \leq \sum_{n \geq 1} \mu(f^{-1}A_n \setminus A_n) = 0$. Por lo tanto $\mu(f^{-1}(\cup_{n \geq 1} A_n) \setminus (\cup_{n \geq 1} A_n)) = 0$. Con esto se concluye que $\Sigma_f$ es un $\sigma$-álgebra. 

\begin{definicion}
	Sea $(X,\Sigma,\mu)$ un espacio de probabilidad. $\mu$ es ergódica si para cualquier $A \in \Sigma$ se tiene	
	\begin{equation}
		\mu A = 0 \vee \mu A = 1
	\end{equation}
\end{definicion}

\begin{definicion}
	Se define $\mathcal{M}(X,\Sigma)$ de la siguiente manera	
	\begin{equation}
		\mathcal{M}(X,\Sigma) = \{\mu: \Sigma \rightarrow \mathbb{R}: \mu \text{ es un medida de probabilidad} \}
	\end{equation}
\end{definicion}

\begin{definicion}
	Se define $\mathcal{M}_f(X,\Sigma)$ de la siguiente manera	
	\begin{equation}
		\mathcal{M}_f(X,\Sigma) = \{ \mu \in \mathcal{M}(X,\Sigma): \mu(f^{-1} A) = \mu A, \forall A \in \Sigma \}
	\end{equation}
\end{definicion}

\begin{definicion}
	Se define $\varepsilon_f(X,\Sigma)$ de la siguiente manera	
	\begin{equation}
		\varepsilon_f(X,\Sigma) = \{ \mu \in \mathcal{M}_f(X,\Sigma): \mu A = 0 \vee \mu A = 1, \forall A \in \Sigma \}
	\end{equation}
\end{definicion}

\begin{definicion}
	Sea $(X,\Sigma,\mu)$ un espacio de probabilidad. Entonces para cualquier conjunto $A \subset \Sigma$, $f_*: \mathcal{M}(X,\Sigma) \rightarrow \mathcal{M}(X,\Sigma)$ se define como	
	\begin{equation}
		f_*\mu(A) = \mu(f^{-1}(A))
	\end{equation}
\end{definicion}

A continuación se analizará la existencia de medidas $f$-invariantes y la topología del conjunto de medidas de probabilidad $f$-invariante una vez que ya se prueba la existencia de alguna medida de ese tipo. Ese análisis se hará a través del desarrollo del Teorema de Krylov-Bugoliubov diviendo la demostración en varios lemas.

\begin{lema}\label{lema1_krylov-bugoliubov}
	Sea $(X,\Sigma,\mu)$ un espacio de probabilidad y sean $\varphi \in \mathcal{C}(X)$. Entonces
	
	\begin{equation}
		\int \varphi d(f_*\mu)=\int \varphi \circ f d\mu
	\end{equation}
\end{lema}

\begin{lema}
	$f_* \in \mathcal{C}(\mathcal{M}(X,\Sigma))$. 
\end{lema}

\begin{lema}\label{lema3_krylov}
	Sean $\varphi \in \mathcal{C}(X)$. $\mu \in \mathcal{M}_f(X,\Sigma)$ si y sólo si
	
	\begin{equation}
		\int \varphi \circ f d\mu = \int \varphi d\mu
	\end{equation}
\end{lema}

\begin{lema}\label{krylov-bugoliubov-lema3}
	$\mathcal{M}(X,\Sigma)$ es compacto. 
\end{lema}

\begin{proof}
	Se demostrará que $\mathcal{M}(X,\Sigma)$ es secuencialmente compacto en la topología débil *.
	
	Se sabe que cualquier conjunto compacto de un espacio métrico es secuencialmente compacto y viceversa. La demostración se puede encontrar en \cite{schep}.
	
	Por otro lado, en \cite{walkden} se puede ver la demostración que $\mathcal{M}(X,\Sigma)$ es un espacio métrico. En conclusión, solo se necesita demostrar que $\mathcal{M}(X,\Sigma)$ es secuencialmente compacto para afirmar que es compacto.
	
	Sea $(\mu_n) \subset \mathcal{M}(X,\Sigma)$ una sucesión. Por otro lado, como $\mathcal{C}(X)$ es separable entonces existe un conjunto denso $\{f_n\}_{n \geq 1} \subset \mathcal{C}(X)$. Esto quiere decir que para cualquier $f \in \mathcal{C}(X)$ se tiene que dando un $\epsilon > 0$ existe $f_i$ tal que
	
	\begin{equation}
		|f - f_i| < \epsilon
	\end{equation}
	
	Definamos
	
	\begin{equation}
		I(\mu,f) := \int f d\mu
	\end{equation}
	
	Para $f_1$ tomemos $I(\mu_n,f_1)$. Se observa que $I(\mu_n,f_1)$ es una sucesión en $\mathbb{R}$. Entonces
	
	\begin{equation}
		|I(\mu_n,f_1)| = \left| \int f_1 d\mu_n \right| \leq \int |f_1| d\mu_n = |f_1|
	\end{equation} 
	
	$I(\mu_n,f_1)$ es una sucesión acotada en $\mathbb{R}$ entonces tiene una subsucesión convergente. Sea el conjunto $\mathbb{N}_1 \subset \mathbb{N}$ el conjunto de los índices de la subsucesión. 
	
	Ahora trabajando con los índices $\mathbb{N}_1$ se puede ver también que $I(\mu_n,f_2)$, para $n \in \mathbb{N}_1$, también es acotada. Entonces se puede tomar una subsucesión convergente con índices $\mathbb{N}_2 \subset \mathbb{N}_1$.
	
	Este proceso se puede seguir hasta conseguir un conjunto de índices $\mathbb{N}_i \subset \mathbb{N}_{i-1} \subset \ldots \subset \mathbb{N}_1 \subset \mathbb{N}$ tal que $I(\mu_n,f_j)$ es convergente para $n \in \mathbb{N}_i$ y para $1 \leq j \leq i$. A está sucesión convergente se le puede extraer una subsucesión $I()$ que seguirá siendo convergente al mismo límite. Sea $I(\mu_{n_i},f_i)$ una subsucesión de las antes nombradas con $n_i \in \mathbb{N}_i$. Como $I(\mu_{n_i},f_i)$ es convergente en $\mathbb{R}$ entonces es una sucesión de Cauchy. Entonces, para el $\epsilon > 0$ que se tomo al inicio, existe $N \in \mathbb{N}$ tal que para $m,k > N$ se cumple
		
	\begin{equation}
		|I(\mu_{n_m},f_i) - I(\mu_{n_k},f_i)| < \epsilon
	\end{equation}
	
	Luego, para el $f$ tomado al inicio, se observa lo siguiente
	
	\begin{align}
		|I(\mu_{n_m},f) - I(\mu_{n_k},f)| &\leq |I(\mu_{n_m},f) - I(\mu_{n_m},f_i) | \\
		& + |I(\mu_{n_m},f_i) - I(\mu_{n_k},f_i)| \\		
		& + |I(\mu_{n_k},f_i) - I(\mu_{n_k},f)|\\
		& < 3\epsilon
	\end{align}
	
	Entonces $I(\mu_{n_i},f)$ es una sucesión de Cauchy en $\mathbb{R}$. Por lo tanto $I(\mu_{n_i},f)$ es convergente.
	
	En resumen, se logró que para cualquier $f \in \mathcal{C}(X)$ existe una subsucesión $(\mu_{n_i})$ tal que
	
	\begin{equation}
		I(\mu_{n_i},f) = \int f d\mu_{n_i}
	\end{equation}
	
	es convergente. Sea $A: \mathcal{C}(X) \rightarrow \mathbb{R}$ tal que esta definida como
	
	\begin{equation}
		Af = \lim_{i \rightarrow \infty} \int f d \mu_{n_i}
	\end{equation}
	
	Se observa que $A$ es lineal y acotado. Entonces se puede aplicar el Teorema de Representación de Riesz \ref{riesz}. En conclusión, existe $\mu \in \mathcal{M}(X,\Sigma)$ tal que
	
	\begin{equation}
		Af = \int f d\mu = \lim_{i \rightarrow \infty} \int f d \mu_{n_i}
	\end{equation}
	
	Por lo tanto $\mathcal{M}(X,\Sigma)$ es secuencialmente compacto con la topología débil *.
\end{proof}

\begin{teorema}[Teorema de Krylov-Bugoliubov]
	$\mathcal{M}_f(X,\Sigma)$ es no vacío, convexo y compacto.
\end{teorema}

\begin{proof}
	Recordar que por el lema \ref{krylov-bugoliubov-lema3} $\mathcal{M}(X,\Sigma)$ es compacto.
	
	Sea $\mu \in \mathcal{M}(X,\Sigma)$, para $n \geq 1$ definimos
	
	\begin{align}
		\mu_n :=& \frac{1}{n} \sum_{i=0}^{n-1} f_*^i \mu
	\end{align}
	
		Sea $\mu_{n_j} \rightarrow \mu$ algún punto límite. Afirmamos que $\mu \in \mathcal{M}_f(X,\Sigma)$, esto quiere decir, $f_*\mu = \mu$.
	
	Por continuidad de $f_*$ se tiene que $f_*\mu_{n_j} \rightarrow f_*\mu$.
	
	Es suficiente demostrar
	
	\begin{equation}
	f_*\mu_{n_j} \rightarrow \mu
	\end{equation}
	
	Veamos
	
	\begin{align}
	f_*\mu_{n_j} =& f_* \left( \frac{1}{n_j} \sum_{i=0}^{n_j-1} f^i_* \mu \right)\\
	=& \frac{1}{n_j} \sum_{i=0}^{n_j-1} f_*^{i+1} \mu\\
	=& \frac{1}{n_j} \left( \sum_{i=0}^{n_j-1} f^i_* \mu - \mu + f^{n_j}_* \mu \right)\\
	=& \frac{1}{n_j} \sum_{i=0}^{n_j-1} f^i_* \mu - \frac{\mu}{n_j} + \frac{f^{n_j}_* \mu}{n_j}\\
	=& \mu_{n_j} - \frac{\mu}{n_j} + \frac{f^{n_j}_* \mu}{n_j}
	\end{align}
	
	Cuando $n_j \rightarrow \infty$ se ve que $f_* \mu_{n_j} \rightarrow \mu$. Por lo tanto $f_*\mu=\mu$. Por lo tanto existe $\mu \in \mathcal{M}_f(X,\Sigma)$ con lo que $\mathcal{M}_f(X,\Sigma) \neq \emptyset$.
	
	Ahora vamos a probar que $\mathcal{M}_f(X,\Sigma)$ es convexo. Sea $\mu_1,\mu_2 \in \mathcal{M}_f(X,\Sigma)$, hay que demostrar que
	
	\begin{equation}
	f_*(t\mu_1 + (1-t)\mu_2) = t\mu_1 + (1-t)\mu_2, \quad t \in [0,1]
	\end{equation}
	
	Veamos. Sea $A \in \Sigma$ cualquiera, luego 
	
	\begin{align}
	f_*(t\mu_1 + (1-t)\mu_2)(A) =& (t\mu_1 + (1-t)\mu_2)(f^{-1}(A))\\
	=& t\mu_1(f^{-1}(A)) + (1-t)\mu_2(f^{-1}(A))\\
	=& t\mu_1(A) + (1-t)\mu_2(A)\\
	=& (t\mu_1 + (1-t)\mu_2)(A)
	\end{align}
	
	Luego $f_*(t\mu_1 + (1-t)\mu_2) = t\mu_1 + (1-t)\mu_2$ con lo que $t\mu_1 + (1-t)\mu_2 \in \mathcal{M}_f(X,\Sigma)$.
	
	Finalmente, probaremos que $\mathcal{M}_f(X,\Sigma)$ es compacto. Como sabemos que $\mathcal{M}(X,\Sigma)$ es compacto porque $M$ es compacto entonces sólo hay que probar que $\mathcal{M}_f(X,\Sigma)$ es cerrado.
	
	Sea $(\mu_n) \subset \mathcal{M}_f(X,\Sigma)$ una sucesión tal que $\mu_n \rightarrow \mu$. Hay que demostrar que $\mu \in \mathcal{M}_f(X,\Sigma)$.
	
	Por definición de convergencia en la topología débil * se tiene que, en particular, para $\varphi \circ f \in \mathcal{C}(X)$ con $\varphi \in \mathcal{C}(X)$ lo siquiente
	
	\begin{align}
	\int \varphi \circ f d\mu &= \lim_{n \rightarrow \infty} \int \varphi \circ f d\mu_n\\
	=& \lim_{n \rightarrow \infty} \int \varphi df_*\mu_n\\
	=& \lim_{n \rightarrow \infty} \int \varphi d\mu_n\\
	=& \int \varphi d\mu
	\end{align}
	
	Por \ref{lema3_krylov} se tiene $\mu \in \mathcal{M}_f(X,\Sigma)$.
\end{proof}

\begin{teorema}
	$\mu \in \varepsilon_f(X,\Sigma)$ si y solo si $\mu$ es un punto extremo de $\mathcal{M}_f(X,\Sigma)$.
\end{teorema}

\begin{proof}
	Supongamos que $\mu \notin \varepsilon_f(X,\Sigma)$. Entonces existe $A \in \Sigma_f$ tal que $\mu A \in (0,1)$. Definimos las medidas de probabilidad $\mu_1,\mu_2$ tales que
	\begin{align}
		\mu_1 B &= \frac{\mu(B \cap A)}{\mu A}\\
		\mu_2 B &= \frac{\mu(B \cap A^c)}{\mu A}
	\end{align}
	
	donde $B \in \Sigma$ cualquiera.
	
	Se observa que $\mu = (\mu A) \mu_1 + (\mu(A^c)) \mu_2$ con $\mu A \in (0,1)$. Entonces $\mu$ no es punto extremo de $\mathcal{M}(X,\Sigma)$ pero lo que se quiere es no sea punto extremo de $\mathcal{M}_f(X,\Sigma)$. Para demostrar la última afirmación se tiene que demostrar que $\mu_1,\mu_2 \in \mathcal{M}_f(X,\Sigma)$. En efecto, sea $B \in \Sigma$ entonces
	\begin{align}
		\mu_1(f^{-1}B) &= \frac{\mu(f^{-1}B \cap A)}{\mu A}\\
		&= \frac{\mu(f^{-1}B \cap f^{-1}A)}{\mu A}\\
		&= \frac{\mu(f^{-1}(B \cap A))}{\mu A}\\
		&= \frac{\mu (B \cap A)}{\mu A}\\		
		&= \mu_1 B
	\end{align}
	
	Entonces $\mu_1 \in \mathcal{M}_f(X,\Sigma)$. Haciendo lo mismo para $\mu_2$ se tiene que $\mu_2 \in \mathcal{M}_f(X,\Sigma)$.
	
	Ahora supongamos que $\mu \in \varepsilon_f(X,\Sigma)$ y supongamos que existen $\mu_1,\mu_2 \in \mathcal{M}_f(X,\Sigma)$ tales que
	\begin{equation}
		\mu = t\mu_1 + (1-t)\mu_2, t \in (0,1)
	\end{equation}
	
	Demostraremos que lo último implica que $\mu_1=\mu_2=\mu$. Vamos a demostrar que $\mu_2=\mu$. Se observa que $\mu_1 \ll \mu$. Luego, por el teorema de Radon-Nykodin \ref{radon-nikodyn}, existe $h_1$ no negativa tal para cualquier $B \in \Sigma$
	\begin{equation}
		\mu_1 B = \int_B h_1 d\mu
	\end{equation}
	
	Si se demuestra que $h_1 = 1$ en c.t.p. entonces tendríamos la equivalencia.
	
	Sean
	\begin{align}
		B &= \{ x \in X: h_1 x < 1 \}\\
		C &= \{ x \in X: h_1 x > 1 \}
	\end{align}
	
	Se observa que
	\begin{align}
		\mu_1 B &= \int_B h_1 d\mu\\
		&= \int_{B \cap f^{-1}B} h_1 d\mu + \int_{B \setminus f^{-1}B} h_1 d\mu\\
	\end{align}
	
	También se observa que
	\begin{align}
		\mu_1(f^{-1}B) &= \int_{f^{-1}B} h_1 d\mu\\
		&= \int_{f^{-1}B \cap B} h_1 d\mu + \int_{f^{-1}B \setminus B} h_1 d\mu\\
	\end{align}
	
	Como $\mu_1 \in \mathcal{M}_f(X,\Sigma)$ entonces $\mu_1 B = \mu_1(f^{-1}B)$.
	
	Luego
	\begin{equation}
		\int_{B \setminus f^{-1}B} h_1 d\mu = \int_{f^{-1}B \setminus B} h_1 d\mu
	\end{equation}
	
	Notar que
	\begin{align}
		\mu(f^{-1}B \setminus B) &= \mu(f^{-1}B) - \mu(f^{-1}B \cap B)\\
		&= \mu B - \mu(f^{-1}B \cap B)\\
		&= \mu(B \setminus f^{-1}B)
	\end{align}
	
	Por la definición de $B$ se tiene
	\begin{align}
		\int_{B \setminus f^{-1}B} h_1 d\mu &< \mu(B \setminus f^{-1}B)\\
		\int_{f^{-1}B \setminus B} h_1 d\mu &\geq \mu(f^{-1}B \setminus B)
	\end{align}
	
	Pero esto no llevaría a una contradicción si la medida es positiva, luego
	\begin{equation}
		\mu(B \setminus f^{-1}B) = \mu(f^{-1}B \setminus B) = 0
	\end{equation}
	
	Así $f^{-1}B=B$ en c.t.p. Luego, por la ergodicidad de $\mu$ se tiene
	\begin{equation}
		\mu B = 1 \vee \mu B = 0
	\end{equation}
	
	Si asumimos que $\mu B = 1$ entonces
	\begin{equation}
		1 = \mu M = \int_M h_1 d\mu = \int_B h_1 d\mu < \mu B = 1
	\end{equation}
	
	Contradicción. Entonces $\mu B = 0$. Procediendo similarmente con $C$ se obtiene $\mu C = 0$. Por lo tanto $h_1 = 1$ en c.t.p.
	
	Por lo tanto $\mu_1=\mu_2=\mu$.
\end{proof}

A continuación se va a demostrar el Teorema Erǵodico de Birkhoff paso a paso a través de una serie de lemas. Este teorema da un corolario que va a a ser útil para dar algunas afirmaciones acerca del estudio de la rotación irracional de $S^1$.

\begin{lema}[Máxima desigualdad]
	Sea $(X,\Sigma,\mu)$ un espacio de probabilidad. $\mu$ es $f$-invariante y $\varphi \in L^1(X,\Sigma,\mu)$. Se define $\varphi_0 = 0$ y para $n \geq 1$
	
	\begin{gather}
		\varphi_n = \sum_{i=0}^{n-1} \varphi \circ f^i\\
		\varPhi_n x = \max \{\varphi_j x : 0 \leq j \leq n\} \geq 0
	\end{gather}
	
	Sea $A = \{x \in X: \varPhi x > 0 \}$ entonces

	\begin{equation}
		\int_A \varphi d\mu \geq 0
	\end{equation}
\end{lema}

\begin{lema}\label{maxima_desigualdad_cor}
	Sea $(X,\Sigma,\mu)$ un espacio de probabilidad y $\varphi \in L^1(X,\Sigma,\mu)$. Si se define
	
	\begin{equation}
		B_{\alpha} = \left\{ x \in X: \sup_{n \geq 1} \frac{1}{n} \sum_{i=0}^{n-1} \varphi \circ f^i x > \alpha \right\}
	\end{equation}
	
	entonces para cualquier $A \in \Sigma_f$ se tiene
	
	\begin{equation}
		\int_{B_{\alpha} \cap A} g d\mu \geq \alpha \mu(B_{\alpha} \cap A)
	\end{equation}
\end{lema}

\begin{lema}\label{birkhoff_lema1}
	Sea $(X,\Sigma,\mu)$ un espacio de probabilidad y $\varphi \in L^1(X,\Sigma,\mu)$. Se definen
	
	\begin{gather}
		h^* x = \limsup_{n \rightarrow \infty} \frac{1}{n} \sum_{i=0}^{n-1} \varphi \circ f^i x\\
		h_* x = \liminf_{n \rightarrow \infty} \frac{1}{n} \sum_{i=0}^{n-1} \varphi \circ f^i x
	\end{gather}
	
	Entonces $h^* = h_*$.
	
\end{lema}

\begin{lema}\label{birkhoff_lema2}
	Sea $h^*$  definido en el lema \ref{birkhoff_lema1}. Entonces $h^* \in L^1(X,\Sigma,\mu)$.
\end{lema}

\begin{lema}\label{birkhoff_lema3}
	Sea $h^*$  definido en el lema \ref{birkhoff_lema1}. Entonces
	
	\begin{equation}
		\int h^* d\mu = \int \varphi d\mu
	\end{equation}
\end{lema}

\begin{proof}
	Para $n \in \mathbb{N}$ y $k \in \mathbb{Z}$ se define
	
	\begin{equation}
		D^n_k = \left\{ x \in X: \frac{k}{n} \leq h^* x \leq \frac{k+1}{n} \right\}
	\end{equation}
	
	Para $\epsilon > 0$ cualquiera tenemos, según el corolario \ref{maxima_desigualdad_cor}, $B_{\frac{k}{n}-\epsilon}$. Es fácil ver que se cumple lo siguiente
	
	\begin{equation}
		D^n_k \cap B_{\frac{k}{n}-\epsilon} = D^n_k
	\end{equation}
	
	Entonces por el corolario \ref{maxima_desigualdad_cor} tenemos
	
	\begin{equation}
		\int_{D^n_k} \varphi d\mu \geq \left( \frac{k}{n} - \epsilon \right) \mu D^n_k
	\end{equation}
	
	Como $\epsilon > 0$ es arbitrario entonces
	
	\begin{equation}
		\int_{D^n_k} \varphi d\mu \geq  \frac{k}{n}  \mu D^n_k	
	\end{equation}
	
	
	Por otro lado, por la definición de $D^n_k$ tenemos
	
	\begin{align}
		\int_{D^n_k} h^* d\mu &\leq \frac{k+1}{n} \mu D^n_k\\
		& \leq \frac{1}{n} \mu D^n_k + \int_{D^n_k} \varphi d\mu
	\end{align}
	
	Puesto que
	
	\begin{equation}
		X = \bigcup_{k \in \mathbb{Z}} D^n_k
	\end{equation}
	
	con los $D^n_k$ disjuntos dos a dos, se puede sumar sobre $\mathbb{Z}$ el resultado anterior. Veamos
	
	\begin{align}
		\sum_{k \in \mathbb{Z}} \int_{D^n_k} h^* d\mu &\leq \frac{1}{n} \sum_{k \in \mathbb{Z}} \mu D^n_k + \sum_{k \in \mathbb{Z}} \int_{D^n_k} \varphi d\mu\\
		\int h^* d\mu &\leq \frac{1}{n} + \int \varphi d\mu
	\end{align}
	
	Esto cumple para cualquier $n \in \mathbb{N}$. Haciendo $n \rightarrow \infty$ se obtiene
	
	\begin{equation}
		\int h^* d\mu \leq \int \varphi d\mu
	\end{equation}
	
	Para completar la demostración reemplazaremos $\varphi$ por $-\varphi$. Con esto la desigualdad anterior generaría a la siguiente desigualdad
	
	\begin{equation}
		\int \varphi d\mu \leq \int h_* d\mu
	\end{equation}
	
	Por lo tanto, como por el lema \ref{birkhoff_lema1} sabemos que $h^*=h_*$ entonces
	
	\begin{equation}
		\int h^* d\mu = \int \varphi d\mu
	\end{equation}
\end{proof}

\begin{teorema}[Teorema Ergódico de Birkhoff]\label{birkhoff_thm}
	Sea $(X,\Sigma,\mu)$ un espacio de probabilidad  con $\mu$ $f$-invariante. Sea $\varphi \in L^1(X,\Sigma,\mu)$ cualquiera y $x \in X$, entonces
	
	\begin{equation}\label{birkhoff_thm_eq1}
		\frac{1}{n} \sum_{i=0}^{n-1} \varphi \circ f^i x \rightarrow E(\varphi,\Sigma_f)
	\end{equation}
	
	para $\mu$-c.t.p.
	
\end{teorema}

\begin{proof}
	Por el lema \ref{birkhoff_lema1} se tiene que $\frac{1}{n} \sum_{i=0}^{n-1} \varphi \circ f^i x$ es convergente.
	
	Por el lema \ref{birkhoff_lema3} se tiene que para cualquier $A \in \Sigma_f$
	
	\begin{equation}
		\int _A h^* d\mu = \int_A f d\mu
	\end{equation}
	
	Además también se puede comprobar que $B_{\frac{k}{n}-\epsilon} \in \Sigma_f$.
	
	Por la definición \ref{esperanza_conficionada_def} de $E(f|\Sigma_f)$ entonces $E(f|\Sigma_f)=h^*$ porque $E(f|\Sigma_f)$ es única.
	
	Por lo tanto
	
	\begin{equation}
		\lim_{n \rightarrow \infty} \frac{1}{n} \sum_{i=0}^{n-1} \varphi \circ f^i x = E(f|\Sigma_f)
	\end{equation}
\end{proof}

\begin{corolario}\label{birkhoff_corolario}
	Sea $(X,\Sigma,\mu)$ un espacio de probabilidad, $\varphi \in L^1(X,\Sigma,\mu)$ y $x \in X$. Si $\mu$ es $f$-invariante y ergódica entonces
	
	\begin{equation}
		\frac{1}{n} \sum_{i=0}^{n-1} \varphi \circ f^i x \rightarrow \int \varphi d\mu
	\end{equation}
	
	cuando $n \rightarrow \infty$ para $\mu$-c.t.p.
\end{corolario}

\begin{proof}
	Por el teorema \ref{birkhoff_thm} se tiene que
	
	\begin{equation}
		\frac{1}{n} \sum_{i=0}^{n-1} \varphi \circ f^i x \rightarrow E(\varphi|N \cap \Sigma_f)
	\end{equation}
	
	Como $N \cap \Sigma_f$ es un sub-$\sigma$-álgebra de $N$ entonces se puede aplicar la propiedad \ref{esperanza_en_sigma-algebra_trivial}. Entonces $E(\varphi|N \cap \Sigma_f) = \int E(\varphi|N \cap \Sigma_f) d\mu = \int \varphi d\mu$. 
\end{proof}