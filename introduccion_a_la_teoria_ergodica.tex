\section{Introducción a la Teoría Ergódica}

En esta sección consideraremos a $X$ como un espacio métrico compacto y a $\Sigma$ como un $\sigma$-álgebra de $X$. 

\begin{definicion}
	Sea $(X,\Sigma,\mu)$ un espacio de probabilidad y $f \in \mathcal{C}(X)$. La medida $\mu$ se dice $f$-invariante si
	
	\begin{equation}
		\mu(f^{-1}A) = \mu A
	\end{equation}
	
	para cualquier $A \in \Sigma$.
\end{definicion}

\begin{definicion}
	Sea $(X,\Sigma,\mu)$ un espacio de probabilidad y $f \in \mathcal{C}(X)$ con $\mu$ $f$-invariante. Se define $\Sigma_f$ como
	
	\begin{equation}
		\Sigma_f = \{ A \in \Sigma: \mu(f^{-1}A) = \mu A \}
	\end{equation}
\end{definicion}

Con esta definición se puede probar $\Sigma_f$ es un $\sigma$-álgebra. Se puede observar que $\emptyset \in \Sigma_f$ por que $\mu(f^{-1} \emptyset) = \mu \emptyset = 0$. Sea $A \in \Sigma_f$. Entonces $\mu(f^{-1}(X \setminus A)) =  \mu(X \setminus f^{-1}A) = 1 - \mu(f^{-1}A) = 1 - \mu A = \mu(X \setminus A)$. Luego $X \setminus A \in \Sigma$. Por último, sea $\{A_n\}_{n \geq 1} \subset \Sigma_f$. Entonces $\mu(f^{-1}A_n \setminus A_n) = 0$. Se sigue que $\mu(f^{-1}(\cup_{n \geq 1} A_n) \setminus (\cup_{n \geq 1} A_n)) \leq \mu(\cup_{n \geq 1} (f^{-1}A_n \setminus A_n)) \leq \sum_{n \geq 1} \mu(f^{-1}A_n \setminus A_n) = 0$. Por lo tanto $\mu(f^{-1}(\cup_{n \geq 1} A_n) \setminus (\cup_{n \geq 1} A_n)) = 0$. Con esto se concluye que $\Sigma_f$ es un $\sigma$-álgebra. 

\begin{definicion}
	Sea $(X,\Sigma,\mu)$ un espacio de probabilidad. $\mu$ es ergódica si para cualquier $A \in \Sigma$ se tiene
	
	\begin{equation}
		\mu A = 0 \vee \mu A = 1
	\end{equation}
\end{definicion}

\begin{definicion}
	Se define $\mathcal{M}(X,\Sigma)$ de la siguiente manera
	
	\begin{equation}
		\mathcal{M}(X,\Sigma) = \{\mu: \Sigma \rightarrow \mathbb{R}: \mu \text{ es un medida de probabilidad} \}
	\end{equation}
\end{definicion}

\begin{definicion}
	Sea $f \in \mathcal{C}(X)$. Se define $\mathcal{M}_f(X,\Sigma)$ de la siguiente manera
	
	\begin{equation}
		\mathcal{M}_f(X,\Sigma) = \{ \mu \in \mathcal{M}(X,\Sigma): \mu(f^{-1} A) = \mu A, \forall A \in \Sigma \}
	\end{equation}
\end{definicion}

\begin{definicion}
	Sea $f \in \mathcal{C}(X)$. Se define $\varepsilon_f(X,\Sigma)$ de la siguiente manera
	
	\begin{equation}
		\varepsilon_f(X,\Sigma) = \{ \mu \in \mathcal{M}_f(X,\Sigma): \mu A = 0 \vee \mu A = 1, \forall A \in \Sigma \}
	\end{equation}
\end{definicion}

\begin{definicion}
	Sea $(X,\Sigma,\mu)$ un espacio de probabilidad, $f \in \mathcal{C}(X)$. Entonces para cualquier conjunto $A \subset \Sigma$, $f_*: \mathcal{M}(X,\Sigma) \rightarrow \mathcal{M}(X,\Sigma)$ se define como
	
	\begin{equation}
		f_*\mu(A) = \mu(f^{-1}(A))
	\end{equation}
\end{definicion}

\begin{lema}\label{lema1_krylov-bugoliubov}
	Sea $(X,\Sigma,\mu)$ un espacio de probabilidad y sean $f,\varphi \in \mathcal{C}(X)$. Entonces
	
	\begin{equation}
		\int \varphi d(f_*\mu)=\int \varphi \circ f d\mu
	\end{equation}
\end{lema}

\begin{lema}
	$f_* \in \mathcal{C}(\mathcal{M}(X,\Sigma))$. 
\end{lema}

\begin{lema}\label{lema3_krylov}
	Sean $f,\varphi \in \mathcal{C}(X)$. $\mu \in \mathcal{M}_f(X,\Sigma)$ si y sólo si
	
	\begin{equation}
		\int \varphi \circ f d\mu = \int \varphi d\mu
	\end{equation}
\end{lema}

\begin{lema}
	$\mathcal{M}(X,\Sigma)$ es compacto. 
\end{lema}

\begin{teorema}[Teorema de Krylov-Bugoliubov]
	Sea $f \in \mathcal{C}(X)$. Entonces $\mathcal{M}_f(X,\Sigma)$ es no vacío, convexo y compacto.
\end{teorema}

\begin{proof}
	Recordar que $\mathcal{M}(X,\Sigma)$ es compacto.
	
	Sea $\mu \in \mathcal{M}(X,\Sigma)$, para $n \geq 1$ definimos
	
	\begin{align}
		\mu_n :=& \frac{1}{n} \sum_{i=0}^{n-1} f_*^i \mu
	\end{align}
	
		Sea $\mu_{n_j} \rightarrow \mu$ algún punto límite. Afirmamos que $\mu \in \mathcal{M}_f(X,\Sigma)$, esto quiere decir, $f_*\mu = \mu$.
	
	Por continuidad de $f_*$ se tiene que $f_*\mu_{n_j} \rightarrow f_*\mu$.
	
	Es suficiente demostrar
	
	\begin{equation}
	f_*\mu_{n_j} \rightarrow \mu
	\end{equation}
	
	Veamos
	
	\begin{align}
	f_*\mu_{n_j} =& f_* \left( \frac{1}{n_j} \sum_{i=0}^{n_j-1} f^i_* \mu \right)\\
	=& \frac{1}{n_j} \sum_{i=0}^{n_j-1} f_*^{i+1} \mu\\
	=& \frac{1}{n_j} \left( \sum_{i=0}^{n_j-1} f^i_* \mu - \mu + f^{n_j}_* \mu \right)\\
	=& \frac{1}{n_j} \sum_{i=0}^{n_j-1} f^i_* \mu - \frac{\mu}{n_j} + \frac{f^{n_j}_* \mu}{n_j}\\
	=& \mu_{n_j} - \frac{\mu}{n_j} + \frac{f^{n_j}_* \mu}{n_j}
	\end{align}
	
	Cuando $n_j \rightarrow \infty$ se ve que $f_* \mu_{n_j} \rightarrow \mu$. Por lo tanto $f_*\mu=\mu$. Por lo tanto existe $\mu \in \mathcal{M}_f(X,\Sigma)$ con lo que $\mathcal{M}_f(X,\Sigma) \neq \emptyset$.
	
	Ahora vamos a probar que $\mathcal{M}_f(X,\Sigma)$ es convexo. Sea $\mu_1,\mu_2 \in \mathcal{M}_f(X,\Sigma)$, hay que demostrar que
	
	\begin{equation}
	f_*(t\mu_1 + (1-t)\mu_2) = t\mu_1 + (1-t)\mu_2, \quad t \in [0,1]
	\end{equation}
	
	Veamos. Sea $A \in \Sigma$ cualquiera, luego 
	
	\begin{align}
	f_*(t\mu_1 + (1-t)\mu_2)(A) =& (t\mu_1 + (1-t)\mu_2)(f^{-1}(A))\\
	=& t\mu_1(f^{-1}(A)) + (1-t)\mu_2(f^{-1}(A))\\
	=& t\mu_1(A) + (1-t)\mu_2(A)\\
	=& (t\mu_1 + (1-t)\mu_2)(A)
	\end{align}
	
	Luego $f_*(t\mu_1 + (1-t)\mu_2) = t\mu_1 + (1-t)\mu_2$ con lo que $t\mu_1 + (1-t)\mu_2 \in \mathcal{M}_f(X,\Sigma)$.
	
	Finalmente, probaremos que $\mathcal{M}_f(X,\Sigma)$ es compacto. Como sabemos que $\mathcal{M}(X,\Sigma)$ es compacto porque $M$ es compacto entonces sólo hay que probar que $\mathcal{M}_f(X,\Sigma)$ es cerrado.
	
	Sea $(\mu_n) \subset \mathcal{M}_f(X,\Sigma)$ una sucesión tal que $\mu_n \rightarrow \mu$. Hay que demostrar que $\mu \in \mathcal{M}_f(X,\Sigma)$.
	
	Por definición de convergencia en la topología débil * se tiene que, en particular, para $\varphi \circ f \in \mathcal{C}(X)$ con $\varphi \in \mathcal{C}(X)$ lo siquiente
	
	\begin{align}
	\int \varphi \circ f d\mu &= \lim_{n \rightarrow \infty} \int \varphi \circ f d\mu_n\\
	=& \lim_{n \rightarrow \infty} \int \varphi df_*\mu_n\\
	=& \lim_{n \rightarrow \infty} \int \varphi d\mu_n\\
	=& \int \varphi d\mu
	\end{align}
	
	Por \ref{lema3_krylov} se tiene $\mu \in \mathcal{M}_f(X,\Sigma)$.
\end{proof}

\begin{teorema}[Máxima desigualdad]
	Sea $(X,\Sigma,\mu)$ un espacio de probabilidad y $f: X \rightarrow X$. $\mu$ es $f$-invariante y $\varphi \in L^1(X,\Sigma,\mu)$. Se define $\varphi_0 = 0$ y para $n \geq 1$
	
	\begin{gather}
		\varphi_n = \sum_{i=0}^{n-1} \varphi \circ f^i\\
		\varPhi_n x = \max \{f_j x : 0 \leq j \leq n\} \geq 0
	\end{gather}
	
	Sea $A = \{x \in X: \varPhi x > 0 \}$ entonces

	\begin{equation}
		\int_A \varphi d\mu \geq 0
	\end{equation}
\end{teorema}

\begin{corolario}
	Sea $(X,\Sigma,\mu)$ un espacio de probabilidad, $f: X \rightarrow X$ y $\varphi \in L^1(X,\Sigma,\mu)$. Si se define
	
	\begin{equation}
		B_{\alpha} = \left\{ x \in X: \sup_{n \geq 1} \frac{1}{n} \sum_{i=0}^{n-1} \varphi \circ f^i x > \alpha \right\}
	\end{equation}
	
	entonces para cualquier $A \in \Sigma_f$ se tiene
	
	\begin{equation}
		\int_{B_{\alpha} \cap A} g d\mu \geq \alpha \mu(B_{\alpha} \cap A)
	\end{equation}
\end{corolario}

\begin{teorema}[Teorema Ergódico de Birkhoff]\label{birkhoff_thm}
	Sea $(X,\Sigma,\mu)$ un espacio de probabilidad y $f: X \rightarrow X$ con $\mu$ $f$-invariante. Sea $\varphi \in L^1(X,\Sigma,\mu)$ cualquiera y $x \in X$, entonces
	
	\begin{equation}
		\frac{1}{n} \sum_{i=0}^{n-1} \varphi \circ f^i x \rightarrow E(\varphi,\Sigma_f)
	\end{equation}
	
	para $\mu$-c.t.p.
	
\end{teorema}

\begin{corolario}\label{birkhoff_corolario}
	Sea $(X,\Sigma,\mu)$ un espacio de probabilidad, $f: X \rightarrow X$, $\varphi \in L^1(X,\Sigma,\mu)$ y $x \in X$. Si $\mu$ es $f$-invariante y ergódica entonces
	
	\begin{equation}
		\frac{1}{n} \sum_{i=0}^{n-1} \varphi \circ f^i x \rightarrow \int \varphi d\mu
	\end{equation}
	
	cuando $n \rightarrow \infty$ para $\mu$-c.t.p.
\end{corolario}

\begin{proof}
	Por el teorema \ref{birkhoff_thm} se tiene que
	
	\begin{equation}
		\frac{1}{n} \sum_{i=0}^{n-1} \varphi \circ f^i x \rightarrow E(\varphi|N \cap \Sigma_f)
	\end{equation}
	
	Como $N \cap \Sigma_f$ es un sub-$\sigma$-álgebra de $N$ entonces se puede aplicar la propiedad \ref{esperanza_en_sigma-algebra_trivial}. Entonces $E(\varphi|N \cap \Sigma_f) = \int E(\varphi|N \cap \Sigma_f) d\mu = \int \varphi d\mu$. 
\end{proof}